\documentclass[a4paper,12pt]{article}
\usepackage[utf8]{inputenc}
\usepackage{amsmath, amssymb, graphicx, geometry, fancyhdr}
\usepackage{listings}
\usepackage{xcolor}

\geometry{margin=2cm}
\pagestyle{fancy}
\fancyhf{}
\fancyhead[L]{Projet - Analyse Numérique}
\fancyfoot[C]{Page \thepage}

\definecolor{codegray}{gray}{0.9}
\lstset{
  backgroundcolor=\color{codegray},
  basicstyle=\ttfamily\footnotesize,
  breaklines=true,
  frame=single,
  numbers=left,
  numberstyle=\tiny,
  stepnumber=1,
  numbersep=5pt,
  showstringspaces=false,
  tabsize=2,
  language=Matlab
}

\begin{document}

\begin{center}
    {\LARGE \textbf{Projet : Estimation numérique d'intégrale}}\\[1em]
    {\large Intégration numérique par méthodes des trapèzes et Simpson}\\[0.5em]
\end{center}

\section*{Introduction}
Certaines intégrales rencontrées en sciences ne possèdent pas de solution analytique simple, 
ou bien la fonction n’est connue que de manière discrète. L’intégration numérique permet d’obtenir 
une approximation contrôlable en fonction du nombre de subdivisions et de la méthode choisie.

Dans ce projet, nous étudions l’intégration de la fonction \(f(x) = e^{-2x}\) sur l’intervalle 
\([0,1]\) en utilisant les méthodes des trapèzes et de Simpson, et comparons les résultats obtenus.  

\section*{Objectifs}
\begin{itemize}
    \item Implémenter les méthodes des trapèzes et Simpson.
    \item Comparer visuellement et numériquement les approximations par rapport à la fonction exacte.
    \item Étudier la convergence et le compromis précision / coût computationnel.
\end{itemize}

\section*{Méthodes numériques}

\subsection*{Méthode des trapèzes}
\[
I \approx \frac{h}{2}\left(f(x_0) + 2f(x_1) + \dots + 2f(x_{n-1}) + f(x_n)\right), 
\quad h = \frac{b-a}{n}.
\]

\subsection*{Méthode de Simpson}
\[
I \approx \frac{h}{3}\left(f(x_0) + 4f(x_1) + 2f(x_2) + 4f(x_3) + \dots + f(x_n)\right), 
\quad n \text{ pair}.
\]
\newpage
\section*{Code MATLAB — Fonctions utilisées}

\subsection*{Fonction trapèzes}
\begin{lstlisting}
function I = trapeze_integration(f, a, b, n)
    h = (b - a) / n;
    x = a:h:b;
    y = f(x);
    I = h * (sum(y(2:end-1)) + (y(1) + y(end))/2);
end
\end{lstlisting}

\subsection*{Fonction Simpson}
\begin{lstlisting}
function I = simpson_integration(f, a, b, n)
    if mod(n, 2) ~= 0
        n = n + 1;
    end
    h = (b - a) / n;
    x = a:h:b;
    y = f(x);
    I = h/3 * (y(1) + 4*sum(y(2:2:end-1)) + 2*sum(y(3:2:end-2)) + y(end));
end
\end{lstlisting}
\newpage
\section*{Tests et visualisations}

\subsection*{Comparaison directe Trapèzes vs Simpson vs fonction exacte}
\begin{lstlisting}
f = @(x) exp(-2*x);
a = 0; b = 1;
n_demo = 10;

h = (b-a)/n_demo;
x_nodes = a:h:b;
y_nodes = f(x_nodes);

I_trap = trapeze_integration(f,a,b,n_demo);
I_simp = simpson_integration(f,a,b,n_demo);

x_fine = linspace(a,b,200);
y_fine = f(x_fine);

figure;
plot(x_fine, y_fine, 'k-', 'LineWidth',2); hold on; grid on;
plot(x_nodes, y_nodes, 'bo-', 'MarkerFaceColor','b');

% Trapezes
for i=1:n_demo
    plot([x_nodes(i) x_nodes(i+1)], [y_nodes(i) y_nodes(i+1)], 'b-', 'LineWidth',1.5);
    fill([x_nodes(i) x_nodes(i) x_nodes(i+1) x_nodes(i+1)], ...
         [0 y_nodes(i) y_nodes(i+1) 0], 'b', 'FaceAlpha',0.2);
end

% Simpson
for i=1:2:n_demo-1
    xx = linspace(x_nodes(i), x_nodes(i+2),50);
    coef = polyfit(x_nodes(i:i+2), y_nodes(i:i+2), 2);
    plot(xx, polyval(coef,xx), 'r-', 'LineWidth',1.5);
end

xlabel('x'); ylabel('f(x)');
title('Comparaison de f(x)=exp(-2x) : Trapèzes vs Simpson');
legend('f(x)','Points','Trapèzes','Simpson');
saveas(gcf,'comparaison_trap_simp.png');
\end{lstlisting}

\section*{Visualisation des figures}
\begin{center}
\includegraphics[width=0.7\textwidth]{comparaison.png}
\end{center}
\vspace{0.5em}
\textit{Figure : Comparaison directe entre Trapèzes, Simpson et la fonction exacte.}

\section*{Analyse des résultats}
\begin{itemize}
    \item \textbf{Précision} : Simpson suit très bien la fonction exacte même pour un nombre de subdivisions faible.
    \item \textbf{Trapèzes} : approximation plus grossière, surtout visible pour peu de subdivisions.
    \item \textbf{Convergence} : Simpson converge beaucoup plus rapidement (ordre 4) que trapèzes (ordre 2).
    \item \textbf{Compromis} : Trapèzes est plus simple et rapide pour peu de précision, Simpson est préférable pour une meilleure précision.
\end{itemize}

\section*{Conclusion}
La comparaison directe montre clairement que la méthode de Simpson est plus performante pour suivre fidèlement 
la fonction \(f(x)=e^{-2x}\). Néanmoins, la méthode des trapèzes reste robuste et simple malgré sa précision.  

Le choix de la méthode dépend du niveau de précision requis et du coût computationnel acceptable.  
L’intégration numérique permet ainsi de traiter efficacement des fonctions sans solution analytique simple.

\end{document}
